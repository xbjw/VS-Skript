\chapter{Einführung}

\begin{stddef} [Transparenz]
	Der User sieht die Komplexität nicht, diese ist durchsichtig.
\end{stddef}

\begin{quote}
	A distributed system is a collection of independent computers that appears to ist users as a single coherent system.
\end{quote}

Middleware ''Verteiltesystemschicht'', sorgt dafür, dass man verteilt bauen kann. Anwendungen sehen unter umständen nur diese Schicht und nicht das Betriebssystem.

\section{Verschiedene Systeme}
\begin{description}
	\item[Kommunikationsverbund] \hfill \\
		Übertragung von Daten bsp. Email.
	\item[Informationsverbund] \hfill \\
		Verbreitung von Informationen, bsp. WWW wie ein Kanal einschalten.
	\item[Datenverbund] \hfill \\
		Speicherung von Daten an verschiedenen Stellen, Datenbanken (DHT), Erhöhung von Verfügbarkeit.
	\item[Lastverbund] \hfill \\
		Aufteilung stoßweiser Lasten, Ressourcen Auslastung gleichschalten. Interessant für Leute, die Stoßzeiten haben, Ticketverkauf. Cloud computing, Leistung bereit stellen.
	\item[Leistungsverbund] \hfill \\
		Anfragen in Teile zerlegen, dadurch schnellere Antworten. Bsp. Wetter wird in einzelnen Quadranten berechnet, um daraus das ''gesammt Wetter'' für Europa berechnet.
	\item[Wartungsverbund] \hfill \\
		Kann Zetral Störungen erkennen und beheben, dadurch Kostenersparrniss, weil nicht jeder Rechner einzelt. Bsp. Pcs für Tägliche veranstaltungen, lassen sich Zentral zurücksetzen.
	\item[Funktionsverbund] \hfill \\
		Spezeille Aufgaben auf spezielle Rechner verteilen, Superrechner, Druckserver.
\end{description}


\section{Wünschenswerte Eigenschaften}
\begin{description}
	\item[Offenheit] Erweiterbarkeit über verschiedene Systeme, im laufenden Betrieb. Schnittstellen zur Verfügung stellen.
	\item[Nebenläufigkeit] Gleichzeitige Prozesse in einem System. Wirklich Parallel nur auf mehreren Prozessoren, Rechnern. Wichtiges Thema ist Synchronisation.
	\item[Skalierbarbeit] funktioniert gut mit wenig und mit vielen Systemen. 
	\item[Sicherheit] Vertraulichkeit, Daten werden von den richtigen gelesen. Integrität, Daten werden unverändert übertragen. Authenzität, Daten stammen von den richtigen Leuten.
	\item[Fehlertoleranz] Fehler ab zu fangen macht ein gutes System aus. Häufige Fehlannahmen:
		\begin{itemize}
			\item Netzwerk ist zuverlässig, sicher und homogen
			\item Topologie ändert sich nicht
			\item Latenzzeit beträgt null
			\item Bandbreite ist unbegrenzt
			\item Energie ist kein Problem (Always-on)
			\item Übertragungskosten betragen null
			\item Empfänger verarbeitet Nachrichten so schnell, wie Sender sendet. (Im Netz kann es zu Staus kommen, keine Garantien.)
			\item Es gibt genau einen Administrator
		\end{itemize}
	\item[Transparenz] Benutzer ist nicht Bewusst, dass er auf einem Verteilten System arbeitet, sieht ein einfacheres Bild.
		\begin{table}[h]
			\begin{tabular}{ l |  p{11cm} }
				Zugriff & Zugriff auf die Ressource erfolgt immer auf die gleiche Art und Weise (lokal oder entfernt) \\ \hline
				Ort & Verbirgt, wo sich eine Ressource befindet. Zugriff über Namen, die keine Ortsinformationen enthalten (Problem: Drucker, Sicherheit) \\ \hline
				Migration & Verschieben von Ressourcen ist für Benutzer und Anwendungen transparent \\ \hline
				Relokation & Verbirgt, dass eine Ressource an einen anderen Ort verschoben werden kann, während sie genutzt wird \\ \hline
				Replikation &  Verbirgt, dass eine Ressource repliziert ist \\ \hline
				Nebenläufigkeit & Verbirgt, dass eine Ressource von mehreren konkurrierenden Benutzern gleichzeitig genutzt werden kann \\ \hline
				Fehler & Verbirgt den Ausfall und die Wiederherstellung einer Ressource \\
				
			\end{tabular}
			\caption{Transparenztypen}
			\label{table:transparenztypen}
		\end{table}
\end{description}


 
\section{Schlusssatz zur Einführung}
Verteilte Systeme bieten gegenüber zentralen einige Vorteilte, sind jedoch komplexer und bedürfen eines sorgfältigen Designs.
