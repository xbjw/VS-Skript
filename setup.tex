%
%
%
% Pakete (fold)
%
%
% Zeichensatz
\usepackage[utf8]{inputenc} % Zeichensatz: UTF 8
\usepackage[T1]{fontenc} % Schriftzeichensatz: europäisch
\usepackage[german]{babel} % Neue Rechtschreibung und Silbentrennung
%
% Farben
\usepackage[dvipsnames,table]{xcolor} % Farben
%
% Seitenlayout
\usepackage{scrpage2}
%
% Grafiken
\usepackage{graphicx}
%
% Titelanpassugnen
\usepackage{titlesec,scalefnt}
%
% Inhaltsanpassungen
\usepackage{titletoc}
%
% Mathematische Pakete
\usepackage[fleqn]{amsmath}
\usepackage{amssymb,amsthm}
%
% Bild- und Teilbildunterschriften (hypcap setzt den Anker vernünftig
\usepackage[hypcap=true]{caption}
\usepackage[hypcap=true]{subcaption}
%
% Listen
\usepackage{enumitem}
\usepackage{tikz}
%
% Tabellen
\usepackage{booktabs,colortbl,longtable}
%
% Quelltext-Einbinden
\usepackage[final]{listings}
%
% Hyperref - load after pagestyle due to captionsredefinition there
\usepackage{url}
\usepackage{hyperref}
\usepackage[all]{hypcap}	% Link to start of figures, not captions
%
% Literatur
\usepackage[numbers,sort]{natbib}
\usepackage{cleveref}
% (end)
%
%
%
% Einstellungen & globale Variablen (fold)
%
% Name, Titel und andere Einstellungen
\newcommand{\UzLTitel}{Verteilte Systeme} %Title
\newcommand{\UzLTitelFormatiert}{%
	\LARGE\color{UzLcolor}\bfseries%
	Verteilte Systeme%
}
% Farbe
\newtheorem{stddef}{Definition}
\definecolor{UzLcolor}{cmyk}{1,.0,.20,.78} % Universitätsfarbe in CMYK
%	\definecolor{UzLcolor}{cmyk}{0,0,0,.9} % In Graustufen diese Alternative wählen
%
% (end)
%
%
%
% Seitenlayout (fold)
%
%
% Seitenlayout setzen
\pagestyle{scrheadings}
\clearscrheadings
%innen: Kapitel Section
\rehead{\leftmark}\lohead{\rightmark}
%außen Seitenzahlen
\rofoot{\pagemark}\lefoot{\pagemark}
%
%
% Schriftart für Überschriften auf serifen, nur bis subsubsection nummerieren
\setkomafont{sectioning}{\rmfamily\color{UzLcolor}\bfseries}
\setkomafont{title}{\color{UzLcolor}}
\setcounter{secnumdepth}{1} % Part/Chapter/Section mit Nummern 
%
% Listen-Einstellungen
\setlist{itemsep=.5\baselineskip,leftmargin=1.5em}
\setlist[enumerate,1]{label=\alph*)}
%
%
% Vermeidung von Hurenkindern und Schusterjungen
\clubpenalty=10000
\widowpenalty=10000
\displaywidowpenalty=10000
%
% Eigene Theorema-Umgebungen
\newtheoremstyle{normalstyle}% ⟨name⟩
{\baselineskip}%	⟨Space above⟩
{}%	⟨Space below⟩
{\normalfont}%	⟨Body font⟩
{}%	⟨Indent amount⟩
{\normalfont\bfseries}% ⟨Theorem head font⟩
{.\newline}%	⟨Punctuation after theorem head⟩
{.5\baselineskip}%	⟨Space after theorem head⟩
{}%	⟨Theorem head spec (can be left empty, meaning ‘normal’)⟩
\theoremstyle{normalstyle}
\newtheorem{thm}{Theorem}[chapter]
\newtheorem{lem}[thm]{Lemma}
\newtheorem{kor}[thm]{Korollar}
\newtheorem{definition}[thm]{Definition}
\newtheorem{rem}[thm]{Bemerkung}
\newtheorem{ex}[thm]{Beispiel}
%
% Abbildungsformatierung
\DeclareCaptionLabelSeparator{periodspace}{.\ }
\captionsetup{format=hang,labelsep=periodspace,indention=-2cm,labelfont=bf,width=.9\textwidth,skip=.5\baselineskip}
\captionsetup[table]{position=above}
\captionsetup[sub]{labelfont=bf,labelsep=period,labelformat=mysublabelfmt,subrefformat=simple}
%
% Optik der Quelltexte mit eigenem Stil
\renewcommand{\lstlistlistingname}{Algorithmenverzeichnis}
\renewcommand{\lstlistingname}{Algorithmus}
\lstset{language=Mathematica}
\lstset{basicstyle={\sffamily\footnotesize},numbers=left,numberstyle=\tiny\color{UzLcolor},numbersep=5pt,
	breaklines=true,
	captionpos={t},
	frame={lines},rulecolor=\color{black},framerule=0.5pt,
	mathescape, columns=flexible,
	tabsize=2
}
\lstdefinestyle{mystyle}
{
	keywordstyle=\bfseries,
	numbers=left,
	numberstyle=\tiny\color{UzLcolor},
	numbersep=5pt,
	breaklines=true,
	frame={lines},
	rulecolor=\color{UzLcolor},
	framerule=0.5pt,
	backgroundcolor=\color{UzLcolor!5},
	aboveskip=1em,
	belowskip=1em,
	showstringspaces=false,
	tabsize=3,
	framesep=0.75em
}	
%
% Chapter Style
\titleformat{\chapter}[display]%
{\relax%
	\raggedleft%
	\huge\bfseries
	\color{UzLcolor}%
}%
{\Huge\raggedleft
	\raggedleft{\textcolor{UzLcolor!25}{\scalefont{8}\thechapter}}}%
{-.5\baselineskip} %
{}
\titlespacing{\chapter}{0pt}{\baselineskip}{\baselineskip}
%
% Design of the table of contents
\titlecontents{chapter}[.5em]{\vspace{1\baselineskip}}{\contentslabel{1em}\large}{\large%\hspace*{-3.2em}
}{\titlerule*[0.5pc]{}{\large\contentspage}}[\vspace{.1\baselineskip}]
\titlecontents{section}[2.5em]{}{\contentslabel{2em}}{%\hspace*{-3.2em}
}{\titlerule*[0.5pc]{.}\contentspage}
\titlecontents{subsection}[3.5em]{}{\contentslabel{1.5em}}{%\hspace*{-3.2em}
}{\titlerule*[0.5pc]{.}\contentspage}
%
\makeatletter% --> Anpassung der Nummernbreite
\renewcommand*{\@pnumwidth}{1.7em}
\makeatother% --> \makeatletter
% (end)
%
%
%
% PDF-Meta-Informationen (fold)
\hypersetup{
	pdftitle={\UzLTitel},
	pdfsubject={Dissertation},
	pdfproducer={pdfLaTeX},
	pdfcreator={YourEditor},
	unicode=true,
	pdfencoding=auto,
	breaklinks=true,
	plainpages=false,
	pdfstartview=FitH,
	pdfview=FitH,
	pdfpagemode=UseOutlines,
	bookmarksnumbered=true,
	bookmarksopen=true,
	bookmarksopenlevel=1,
	pdfdisplaydoctitle=true,
	pdfduplex=true,
	pdflang=de,
	colorlinks=true,
	linkcolor=UzLcolor,
	citecolor=black,
	urlcolor=black
}
% (end)