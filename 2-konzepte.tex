\chapter{Hardware und Software Konzepte}

Hardware und Software Konzepte zur Realisierung von Verteilten Systemen\cite[p.~25]{VS1}

\section{Hardware Konzepte}

Laufen auf Systemen mit mehreren CPUs. Einteilung in:
\begin{description}
	\item[Multiprozessorsyssteme] \hfill \\
		Parallelrechner, mehrere CPUs in einem Rechner. Teuer, da Spezialarchitektur und heute sind praktisch alle Spezialfirmen pleite.
	\item[Multicomputersysteme] \hfill \\
		\begin{description}
			\item[Homogen] alle Rechner sind gleich. Bsp. Cluster von gleichartigen PCs.
			\item[Heterogen] mit sehr unterschiedlichen Architekturen. Bsp. Anwendungen übers Internet. Worauf sich diese vorlesung konzentriert.
		\end{description}
\end{description}

\section{Software Konzepte}

\begin{description}
	\item[Verteilte Betriebssysteme] Betriebssystem für Multi-Prozessor und homogene Multi-Computer. Versteckt und managed Hardware resourcen. (Distributed Operating System DOS).
	\item[Netzwerkbetriebssysteme] Lose gekoppeltes Betriebssystem für heterogene Multi-Computer wie LAN. Bietet lokale Dienste für entfernte Clients. (Network Operating System NOS).
	\item[Middleware] Schicht über NOS, die allgemeine Dienste implementiert. Stellt eine verteilte Transparenz bereit.
\end{description}
