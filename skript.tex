%
%	Studentisches Skript für Verteilte Systeme.

%	Nutzt ein Design von Ronny Bermann https://github.com/kellertuer/UzL-Design/blob/master/Dissertation/dissertation.tex

\documentclass[
cleardoublepage=empty,	% leere Seite beim Kapitel komplett leer
fontsize=11pt,			% Schriftgröße: 11pt
a4paper,				% Format: DIN A4
toc=bibliography,		% Literatur ins Inhaltsverzeichnis
listof=leveldown,		% Einbinden von List of Figures/List of Tables ins Inhaltsverzeichnis
twoside,				% Zweiseitiges Layout
BCOR=13mm,				% Bundstegkorrektur
headinclude=true,		% Kopfzeile gehört zur Texthöhe
footinclude=false,		% ...Fußzeile nicht
parskip=half,			% zwischen zwei Absätzen eine halbe Zeilenhöhe Platz
DIV=10					% Typearea: Seiten in 10x10 teilen, um den Satzspiegel zu bestimmen
%,draft					% für schnellere Kompilierung das Kommentar entfernen, dann keine Grafiken 
,openany
]{scrbook}				% KOMA-Skript-Buchklasse

%Ganz viel schön machen Einbauen.
%
%
%
% Pakete (fold)
%
%
% Zeichensatz
\usepackage[utf8]{inputenc} % Zeichensatz: UTF 8
\usepackage[T1]{fontenc} % Schriftzeichensatz: europäisch
\usepackage[german]{babel} % Neue Rechtschreibung und Silbentrennung
%
% Farben
\usepackage[dvipsnames,table]{xcolor} % Farben
%
% Seitenlayout
\usepackage{scrpage2}
%
% Grafiken
\usepackage{graphicx}
%
% Titelanpassugnen
\usepackage{titlesec,scalefnt}
%
% Inhaltsanpassungen
\usepackage{titletoc}
%
% Mathematische Pakete
\usepackage[fleqn]{amsmath}
\usepackage{amssymb,amsthm}
%
% Bild- und Teilbildunterschriften (hypcap setzt den Anker vernünftig
\usepackage[hypcap=true]{caption}
\usepackage[hypcap=true]{subcaption}
%
% Listen
\usepackage{enumitem}
\usepackage{tikz}
%
% Tabellen
\usepackage{booktabs,colortbl,longtable}
%
% Quelltext-Einbinden
\usepackage[final]{listings}
%
% Hyperref - load after pagestyle due to captionsredefinition there
\usepackage{url}
\usepackage{hyperref}
\usepackage[all]{hypcap}	% Link to start of figures, not captions
%
% Literatur
\usepackage[numbers,sort]{natbib}
\usepackage{cleveref}
% (end)
%
%
%
% Einstellungen & globale Variablen (fold)
%
% Name, Titel und andere Einstellungen
\newcommand{\UzLTitel}{Verteilte Systeme} %Title
\newcommand{\UzLTitelFormatiert}{%
	\LARGE\color{UzLcolor}\bfseries%
	Verteilte Systeme%
}
% Farbe
\newtheorem{stddef}{Definition}
\definecolor{UzLcolor}{cmyk}{1,.0,.20,.78} % Universitätsfarbe in CMYK
%	\definecolor{UzLcolor}{cmyk}{0,0,0,.9} % In Graustufen diese Alternative wählen
%
% (end)
%
%
%
% Seitenlayout (fold)
%
%
% Seitenlayout setzen
\pagestyle{scrheadings}
\clearscrheadings
%innen: Kapitel Section
\rehead{\leftmark}\lohead{\rightmark}
%außen Seitenzahlen
\rofoot{\pagemark}\lefoot{\pagemark}
%
%
% Schriftart für Überschriften auf serifen, nur bis subsubsection nummerieren
\setkomafont{sectioning}{\rmfamily\color{UzLcolor}\bfseries}
\setkomafont{title}{\color{UzLcolor}}
\setcounter{secnumdepth}{1} % Part/Chapter/Section mit Nummern 
%
% Listen-Einstellungen
\setlist{itemsep=.5\baselineskip,leftmargin=1.5em}
\setlist[enumerate,1]{label=\alph*)}
%
%
% Vermeidung von Hurenkindern und Schusterjungen
\clubpenalty=10000
\widowpenalty=10000
\displaywidowpenalty=10000
%
% Eigene Theorema-Umgebungen
\newtheoremstyle{normalstyle}% ⟨name⟩
{\baselineskip}%	⟨Space above⟩
{}%	⟨Space below⟩
{\normalfont}%	⟨Body font⟩
{}%	⟨Indent amount⟩
{\normalfont\bfseries}% ⟨Theorem head font⟩
{.\newline}%	⟨Punctuation after theorem head⟩
{.5\baselineskip}%	⟨Space after theorem head⟩
{}%	⟨Theorem head spec (can be left empty, meaning ‘normal’)⟩
\theoremstyle{normalstyle}
\newtheorem{thm}{Theorem}[chapter]
\newtheorem{lem}[thm]{Lemma}
\newtheorem{kor}[thm]{Korollar}
\newtheorem{definition}[thm]{Definition}
\newtheorem{rem}[thm]{Bemerkung}
\newtheorem{ex}[thm]{Beispiel}
%
% Abbildungsformatierung
\DeclareCaptionLabelSeparator{periodspace}{.\ }
\captionsetup{format=hang,labelsep=periodspace,indention=-2cm,labelfont=bf,width=.9\textwidth,skip=.5\baselineskip}
\captionsetup[table]{position=above}
\captionsetup[sub]{labelfont=bf,labelsep=period,labelformat=mysublabelfmt,subrefformat=simple}
%
% Optik der Quelltexte mit eigenem Stil
\renewcommand{\lstlistlistingname}{Algorithmenverzeichnis}
\renewcommand{\lstlistingname}{Algorithmus}
\lstset{language=Mathematica}
\lstset{basicstyle={\sffamily\footnotesize},numbers=left,numberstyle=\tiny\color{UzLcolor},numbersep=5pt,
	breaklines=true,
	captionpos={t},
	frame={lines},rulecolor=\color{black},framerule=0.5pt,
	mathescape, columns=flexible,
	tabsize=2
}
\lstdefinestyle{mystyle}
{
	keywordstyle=\bfseries,
	numbers=left,
	numberstyle=\tiny\color{UzLcolor},
	numbersep=5pt,
	breaklines=true,
	frame={lines},
	rulecolor=\color{UzLcolor},
	framerule=0.5pt,
	backgroundcolor=\color{UzLcolor!5},
	aboveskip=1em,
	belowskip=1em,
	showstringspaces=false,
	tabsize=3,
	framesep=0.75em
}	
%
% Chapter Style
\titleformat{\chapter}[display]%
{\relax%
	\raggedleft%
	\huge\bfseries
	\color{UzLcolor}%
}%
{\Huge\raggedleft
	\raggedleft{\textcolor{UzLcolor!25}{\scalefont{8}\thechapter}}}%
{-.5\baselineskip} %
{}
\titlespacing{\chapter}{0pt}{\baselineskip}{\baselineskip}
%
% Design of the table of contents
\titlecontents{chapter}[.5em]{\vspace{1\baselineskip}}{\contentslabel{1em}\large}{\large%\hspace*{-3.2em}
}{\titlerule*[0.5pc]{}{\large\contentspage}}[\vspace{.1\baselineskip}]
\titlecontents{section}[2.5em]{}{\contentslabel{2em}}{%\hspace*{-3.2em}
}{\titlerule*[0.5pc]{.}\contentspage}
\titlecontents{subsection}[3.5em]{}{\contentslabel{1.5em}}{%\hspace*{-3.2em}
}{\titlerule*[0.5pc]{.}\contentspage}
%
\makeatletter% --> Anpassung der Nummernbreite
\renewcommand*{\@pnumwidth}{1.7em}
\makeatother% --> \makeatletter
% (end)
%
%
%
% PDF-Meta-Informationen (fold)
\hypersetup{
	pdftitle={\UzLTitel},
	pdfsubject={Dissertation},
	pdfproducer={pdfLaTeX},
	pdfcreator={YourEditor},
	unicode=true,
	pdfencoding=auto,
	breaklinks=true,
	plainpages=false,
	pdfstartview=FitH,
	pdfview=FitH,
	pdfpagemode=UseOutlines,
	bookmarksnumbered=true,
	bookmarksopen=true,
	bookmarksopenlevel=1,
	pdfdisplaydoctitle=true,
	pdfduplex=true,
	pdflang=de,
	colorlinks=true,
	linkcolor=UzLcolor,
	citecolor=black,
	urlcolor=black
}
% (end)


% Dokumentanfang 
\begin{document}
	
	
	% Abschnitt vor dem eigentlichen Text
	\frontmatter
	
	% Titelseite
	\begin{titlepage}
		\parskip=0pt
		\parindent=0pt
		% Notwendig: Eigenes Institutslogo
	%	\includegraphics[width=91mm]{logos/institutslogo.eps}
		\par
		\vspace{6\baselineskip}
		\par\vspace{\baselineskip}
		{\UzLTitelFormatiert}
		\par
		\vspace{\baselineskip}
		Studentische Mitschrift als Skript.
		\par
		
	\end{titlepage}

	% Inhaltsverzeichnis
	\tableofcontents

	% Hauptteil (fold)
	\mainmatter
	
	% %Einzelne Kapitel hier einbinden.	
	\chapter{Einführung}

\begin{stddef} [Transparenz]
	Der User sieht die Komplexität nicht, diese ist durchsichtig.
\end{stddef}

\begin{quote}
	A distributed system is a collection of independent computers that appears to ist users as a single coherent system.
\end{quote}

Middleware ''Verteiltesystemschicht'', sorgt dafür, dass man verteilt bauen kann. Anwendungen sehen unter umständen nur diese Schicht und nicht das Betriebssystem.

\section{Verschiedene Systeme}
\begin{description}
	\item[Kommunikationsverbund] \hfill \\
		Übertragung von Daten bsp. Email.
	\item[Informationsverbund] \hfill \\
		Verbreitung von Informationen, bsp. WWW wie ein Kanal einschalten.
	\item[Datenverbund] \hfill \\
		Speicherung von Daten an verschiedenen Stellen, Datenbanken (DHT), Erhöhung von Verfügbarkeit.
	\item[Lastverbund] \hfill \\
		Aufteilung stoßweiser Lasten, Ressourcen Auslastung gleichschalten. Interessant für Leute, die Stoßzeiten haben, Ticketverkauf. Cloud computing, Leistung bereit stellen.
	\item[Leistungsverbund] \hfill \\
		Anfragen in Teile zerlegen, dadurch schnellere Antworten. Bsp. Wetter wird in einzelnen Quadranten berechnet, um daraus das ''gesammt Wetter'' für Europa berechnet.
	\item[Wartungsverbund] \hfill \\
		Kann Zetral Störungen erkennen und beheben, dadurch Kostenersparrniss, weil nicht jeder Rechner einzelt. Bsp. Pcs für Tägliche veranstaltungen, lassen sich Zentral zurücksetzen.
	\item[Funktionsverbund] \hfill \\
		Spezeille Aufgaben auf spezielle Rechner verteilen, Superrechner, Druckserver.
\end{description}


\section{Wünschenswerte Eigenschaften}
\begin{description}
	\item[Offenheit] Erweiterbarkeit über verschiedene Systeme, im laufenden Betrieb. Schnittstellen zur Verfügung stellen.
	\item[Nebenläufigkeit] Gleichzeitige Prozesse in einem System. Wirklich Parallel nur auf mehreren Prozessoren, Rechnern. Wichtiges Thema ist Synchronisation.
	\item[Skalierbarbeit] funktioniert gut mit wenig und mit vielen Systemen. 
	\item[Sicherheit] Vertraulichkeit, Daten werden von den richtigen gelesen. Integrität, Daten werden unverändert übertragen. Authenzität, Daten stammen von den richtigen Leuten.
	\item[Fehlertoleranz] Fehler ab zu fangen macht ein gutes System aus. Häufige Fehlannahmen:
		\begin{itemize}
			\item Netzwerk ist zuverlässig, sicher und homogen
			\item Topologie ändert sich nicht
			\item Latenzzeit beträgt null
			\item Bandbreite ist unbegrenzt
			\item Energie ist kein Problem (Always-on)
			\item Übertragungskosten betragen null
			\item Empfänger verarbeitet Nachrichten so schnell, wie Sender sendet. (Im Netz kann es zu Staus kommen, keine Garantien.)
			\item Es gibt genau einen Administrator
		\end{itemize}
	\item[Transparenz] Benutzer ist nicht Bewusst, dass er auf einem Verteilten System arbeitet, sieht ein einfacheres Bild.
		\begin{table}[h]
			\begin{tabular}{ l |  p{11cm} }
				Zugriff & Zugriff auf die Ressource erfolgt immer auf die gleiche Art und Weise (lokal oder entfernt) \\ \hline
				Ort & Verbirgt, wo sich eine Ressource befindet. Zugriff über Namen, die keine Ortsinformationen enthalten (Problem: Drucker, Sicherheit) \\ \hline
				Migration & Verschieben von Ressourcen ist für Benutzer und Anwendungen transparent \\ \hline
				Relokation & Verbirgt, dass eine Ressource an einen anderen Ort verschoben werden kann, während sie genutzt wird \\ \hline
				Replikation &  Verbirgt, dass eine Ressource repliziert ist \\ \hline
				Nebenläufigkeit & Verbirgt, dass eine Ressource von mehreren konkurrierenden Benutzern gleichzeitig genutzt werden kann \\ \hline
				Fehler & Verbirgt den Ausfall und die Wiederherstellung einer Ressource \\
				
			\end{tabular}
			\caption{Transparenztypen}
			\label{table:transparenztypen}
		\end{table}
\end{description}


 
\section{Schlusssatz zur Einführung}
Verteilte Systeme bieten gegenüber zentralen einige Vorteilte, sind jedoch komplexer und bedürfen eines sorgfältigen Designs.

	\chapter{Hardware und Software Konzepte}

Hardware und Software Konzepte zur Realisierung von Verteilten Systemen\cite[p.~25]{VS1}

\section{Hardware Konzepte}

Laufen auf Systemen mit mehreren CPUs. Einteilung in:
\begin{description}
	\item[Multiprozessorsyssteme] \hfill \\
		Parallelrechner, mehrere CPUs in einem Rechner. Teuer, da Spezialarchitektur und heute sind praktisch alle Spezialfirmen pleite.
	\item[Multicomputersysteme] \hfill \\
		\begin{description}
			\item[Homogen] alle Rechner sind gleich. Bsp. Cluster von gleichartigen PCs.
			\item[Heterogen] mit sehr unterschiedlichen Architekturen. Bsp. Anwendungen übers Internet. Worauf sich diese vorlesung konzentriert.
		\end{description}
\end{description}

\section{Software Konzepte}

\begin{description}
	\item[Verteilte Betriebssysteme] Betriebssystem für Multi-Prozessor und homogene Multi-Computer. Versteckt und managed Hardware resourcen. (Distributed Operating System DOS).
	\item[Netzwerkbetriebssysteme] Lose gekoppeltes Betriebssystem für heterogene Multi-Computer wie LAN. Bietet lokale Dienste für entfernte Clients. (Network Operating System NOS).
	\item[Middleware] Schicht über NOS, die allgemeine Dienste implementiert. Stellt eine verteilte Transparenz bereit.
\end{description}


	% Bibliothek
	\bibliographystyle{abbrv}
	\bibliography{bibliography}
	
	\begingroup
	\let\chapter=\section
	
	\listoffigures %Abbildungsverzeichnis
	\lstlistoflistings %Quelltextverzeichnis
	\listoftables %Tabellenverzechnis
	% Add further lists here
	\endgroup
	\appendix
\end{document}